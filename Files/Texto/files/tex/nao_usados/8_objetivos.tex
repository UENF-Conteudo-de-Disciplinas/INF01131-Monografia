\chapter{Objetivos}

Como objetivos gerais, espera-se conseguir desenvolver um sistema de suporte à decisão tal que aumente a eficiência e eficácia do processo de criação de grades horárias que semestralmente demandam extensa quantidade de tempo dos coordenadores de curso na UENF. Nesse processo, também é esperado que as grades horárias finais tragam também benefícios aos alunos. Visto que estes muitas vezes lidam com grades horárias que não contemplam suas reais demandas.

Como objetivos mais específicos, podemos listar os seguintes:

\begin{enumerate}
    \item Entender de que forma os setores administrativos da UENF atualmente lidam com a questão do \textit{timetabling}
    \item Obter as demandas de aprimoramentos desejadas pelos diferentes centros e laboratórios
    \item Modelar o sistema de resolução de \textit{timetabling} de acordo com os requisitos demandados
    \item Encontrar o que é necessário para a adoção da aplicação de tabelamento de horário
    \item Incentivar o uso de uma ferramenta centralizada para a otimização do \textit{Timetabling Problem}
\end{enumerate}
