\chapter{Conclusões} \label{chap:conclusoes}

% CONCLUIR CAPÍTULO POR CAPÍTULO

O problema de organização de grade horária no ensino superior tem sido amplamente estudado por diversos pesquisadores. Devido sua natureza multidimensional e com forte tendência a especificidades, este campo de estudo se mostra como amplo e complexo.

Através da revisão bibliográfica, foi possível observar que a maioria dos trabalhos se foca em um método heurístico de solução, onde se busca uma solução ótima, ou próxima do ótimo, através de um método de busca. Entretanto, o presente trabalho se propõe a uma abordagem diferente, onde se busca uma solução boa o suficiente para que seja utilizada na prática, mesmo que não seja ótima, isso através do método de manipulação manual dos dados.

Para este fim foi desenvolvido um sistema de suporte à decisão para auxiliar os setores da Universidade Estadual do Norte Fluminense Darcy Ribeiro (UENF) responsáveis pela criação de grades horárias. O sistema foi desenvolvido com o intuito de ser utilizado como uma ferramenta auxiliar, onde os usuários possam manipular os dados de forma mais intuitiva e visual, assim reduzindo a necessidade de retrabalho e aumentando a produtividade.

O sistema permite que as quatro operações básicas de armazenamento persistente, sendo elas a criação, leitura, edição e deleção de dados. Com isso, os usuários podem adicionar manualmente as informações referentes ao trabalho de criação de grade horária de forma centralizada, assim reduzindo a necessidade de se lidar com diversos arquivos e planilhas. Facilitando também a visualização de informações, como a alocação de turmas, que pode ser visualizada de forma gráfica, assim facilitando a identificação de conflitos e problemas. O que consequentemente tende a agilizar o processo de busca por novas soluções e a redução dos conflitos.

% Tendo atingido este objetivo, espera-se que o sistema possa ser utilizado na prática, e que possa ser aprimorado e melhorado com o tempo.


\section{Alternativas Burocráticas} % ### 7.2.

Além da busca pela solução ótima, o presente trabalho também se propõe a buscar métodos ainda mais alternativos para se amenizar a problemática abordada. Sendo, de forma simples, o uso de meios burocráticos disponíveis na instituição que abre alguns caminhos para uma maior praticidade no processo de resolução do problema. Entretanto, é necessário que se tenha em mente que a burocracia é um processo lento e que pode ser desgastante, sendo até mesmo esperado que não seja desejado por parte dos construtores da grade horária.

Essas alternativas não geram por si só uma solução para o problema, em termos metafóricos, se o sistema é a engrenagem que faz a máquina funcionar, as alternativas burocráticas são o óleo que pode fazer a engrenagem funcionar de forma mais suave, mesmo que não seja estritamente necessário.

\subsection{Tempo de elaboração das grades} \label{subsec:burocracia-férias} % #### 7.1.1.

Durante as entrevistas do \autoref{chap:instituicao} da \autoref{sec:entrevistas}, uma alternativa válida para a amenização da problemática abordada é a alteração do calendário anual da UENF que define férias de duas semanas entre os semestres. Caso seu calendário seja alterado para que as férias sejam de três semanas, o problema de agendamento teria maior tempo para ser resolvido, assim fazendo com que a solução ótima seja provável de ser alcançada.

% Citar o Estatuto da forma correta.

Segundo o Artigo 28 do Estatuto da UENF, compete à secretaria acadêmica a elaboração da proposta de calendário escolar para que seja aprovado pelo Colegiado Acadêmico. Enquanto que o Artigo 63 da seção 2 do capítulo 1, informa que os calendários do curso de graduação devem ser aprovados pelas correspondentes câmaras, com observância do calendário da universidade.

Logo, quanto à alteração do calendário acadêmico, a alteração mostra-se como possível, sendo necessário apenas que o processo burocrático necessário seja enfrentado, o que pode acabar não sendo do desejo daqueles que constroem a grade horária.

\subsection{Alteração forçada de horários} \label{subsec:burocracia-troca} % #### 7.1.2.

Segundo o parágrafo primeiro do artigo 36 das Normas de Graduação, ``qualquer alteração de horário/turno após o período de matrícula deverá ter a anuência por escrito de todos os discentes matriculados na turma''. Seguindo ao segundo parágrafo do mesmo artigo, temos que ``a alteração de horário das aulas da turma deverá ter a anuência da Coordenação de Curso e a ciência do Chefe do Laboratório responsável pela disciplina''.

Outra alternativa que aproveita de uma brecha nas normas supracitadas é a possibilidade de se criar novas turmas para as disciplinas que possuem horários conflitantes, assim direcionando os alunos para que se desinscrevam da turma anterior.

Ambas as alternativas supracitadas visam a alteração dos horários das turmas mesmo após o estimado período de construção da grade horária, assim efetivamente aumentando novamente o tempo disponível para a resolução do problema.

\subsection{Aplicação prática dos métodos burocráticos} % #### 7.1.3.

% O QUE DEVERIA TER INTERVALO É ENTRE O FIM DAS INSCRIÇÕES E O INÍCIO DAS AULAS

% adicionar também a possibilidade da mudança informal do horário

Consideremos o ano de 2023, os seus semestres e seus respectivos calendário acadêmicos para a graduação, onde a \autoref{tab:calendario_SECACAD-2023.1} mostra o calendário do primeiro semestre e a \autoref{tab:calendario_SECACAD-2023.2} mostra o calendário do segundo semestre.

% Abreviados
\begin{table}[H] \centering \caption{Calendário Acadêmico da SECACAD de 2023.1 (simplificado)} \label{tab:calendario_SECACAD-2023.1}
  \begin{tabular}{| l r |}
    \hline
    \textbf{Atividades}                                              & \textbf{Data} \\
    \hline
    Prazo limite de cad. de nov. discip. a serem ofer. em 2023.1     & até 20/01     \\
    Prazo limite para criação de turmas a serem oferecidas em 2023.1 & 20/01 a 15/02 \\
    Renovação de matrícula de 2023.1                                 & 28/02 a 03/03 \\
    Início do período letivo de 2023.1                               & 06/03         \\
    Inclusão e exclusão de disciplinas                               & 06/03 à 20/03 \\
    Encerramento do período letivo de 2023.1                         & 07/07         \\
    Prazo limite: Entrega dos resultados à SECACAD                   & 14/07         \\
    \hline
  \end{tabular}
\end{table}

% as is
\begin{comment}
\begin{table}[H] \centering
  \caption{Calendário Acadêmico da SECACAD de 2023.1 (simplificado)}
  \label{tab:calendario_SECACAD-2023.1}
  \begin{tabular}{| l r |}
    \hline
    \textbf{Atividades}                                                                           & \textbf{Data}           \\
    \hline
    Prazo limite de cadastro de novas disciplinas a serem oferecidas no 1º período letivo de 2023 & até 20/01/2023          \\
    Prazo limite para criação de turmas a serem oferecidas no 1º período letivo de 2023           & 20/01/2023 a 15/02/2023 \\
    Renovação de matrícula do 1º período letivo/2023                                              & 28/02 a 03/03           \\
    Início do 1º período letivo de 2023                                                           & 06/03                   \\
    Inclusão e exclusão de disciplinas                                                            & 06/03 à 20/03           \\
    Encerramento do 1º período letivo de 2023                                                     & 07/07                   \\
    \hline
  \end{tabular}
\end{table}

\begin{table}[H] \centering
  \caption{Calendário Acadêmico da SECACAD de 2023.2 (simplificado)}
  \label{tab:calendario_SECACAD-2023.2}
  \begin{tabular}{| l r |}
    \hline
    \textbf{Atividades}                                                                           & \textbf{Data} \\
    \hline
    Prazo limite de cadastro de novas disciplinas a serem oferecidas no 2º período letivo de 2023 & até 14/07     \\
    Prazo limite: Entrega dos resultados à SECACAD                                                & 14/07         \\
    Prazo limite para criação de turmas a serem oferecidas no 2º período letivo de 2023           & 17 a 28/07    \\
    Renovação de matrícula do 2º período letivo/2023                                              & 01/08 a 04/08 \\
    Início do 2º período letivo de 2023                                                           & 07/08         \\
    Inclusão e exclusão de disciplinas                                                            & 14 a 21/08    \\
    Encerramento do 2º período letivo de 2023                                                     & 08/12         \\
    Prazo limite: Entrega dos resultados à SECACAD                                                & 15/12         \\
    \hline
  \end{tabular}
\end{table}
\end{comment}

\begin{table}[H] \centering \caption{Calendário Acadêmico da SECACAD de 2023.2 (simplificado)} \label{tab:calendario_SECACAD-2023.2}
  \begin{tabular}{| l r |}
    \hline
    \textbf{Atividades}                                              & \textbf{Data} \\
    \hline
    Prazo limite de cad. de nov. discip. a serem ofer. em 2023.2     & até 14/07     \\
    Prazo limite para criação de turmas a serem oferecidas em 2023.2 & 17 a 28/07    \\
    Renovação de matrícula de 2023.2                                 & 01/08 a 04/08 \\
    Início do período letivo de 2023.2                               & 07/08         \\
    Inclusão e exclusão de disciplinas                               & 14 a 21/08    \\
    Encerramento do período letivo de 2023.2                         & 08/12         \\
    Prazo limite: Entrega dos resultados à SECACAD                   & 15/12         \\
    \hline
  \end{tabular}
\end{table}

Com objetivo de tormar a visualização das informações mais coesa, a \autoref{tab:calendario_2023-Enxuto} mostra um calendário acadêmico normalizado, onde os prazos, períodos e marcações de início e fim foram convertidos em termos de ``início'' e ``fim'', assim distinguindo também o seu período de vigência e a data específica.

\begin{table}[H] \centering \caption{Calendário Acadêmico de 2023 (simplificado)} \label{tab:calendario_2023-Enxuto}
  \begin{tabular}{| c r l c |}
    \hline
    \textbf{Vigência} & \textbf{Fase}         & \textbf{Atividades}                              & \textbf{Data} \\
    \hline
    % 2022.2          & \textbf{Fim}          & período letivo                                   & 14/12/22      \\
    % 2022.2          & \textbf{\textit{Fim}} & entrega dos resultados à SECACAD                 & 21/12/22      \\

    2023.1            & \textbf{\textit{Fim}} & cadastro de novas disciplinas a serem oferecidas & 20/01/23      \\
    2023.1            & \textbf{Início}       & criação de turmas a serem oferecidas             & 20/01/23      \\
    2023.1            & \textbf{Fim}          & criação de turmas a serem oferecidas             & 15/02/23      \\
    2023.1            & \textbf{Início}       & renovação de matrícula                           & 28/02/23      \\
    2023.1            & \textbf{Fim}          & renovação de matrícula                           & 03/03/23      \\
    2023.1            & \textbf{Início}       & período letivo                                   & 06/03/23      \\
    2023.1            & \textbf{Início}       & inclusão e exclusão de disciplinas               & 06/03/23      \\
    2023.1            & \textbf{Fim}          & inclusão e exclusão de disciplinas               & 20/03/23      \\
    2023.1            & \textbf{Fim}          & período letivo                                   & 07/07/23      \\
    2023.1            & \textbf{\textit{Fim}} & entrega dos resultados à SECACAD                 & 14/07/23      \\

    2023.2            & \textbf{\textit{Fim}} & cadastro de novas disciplinas a serem oferecidas & 14/07/23      \\
    2023.2            & \textbf{Início}       & criação de turmas a serem oferecidas             & 17/07/23      \\
    2023.2            & \textbf{Fim}          & criação de turmas a serem oferecidas             & 28/07/23      \\
    2023.2            & \textbf{Início}       & renovação de matrícula                           & 01/08/23      \\
    2023.2            & \textbf{Fim}          & renovação de matrícula                           & 04/08/23      \\
    2023.2            & \textbf{Início}       & período letivo                                   & 07/08/23      \\
    2023.2            & \textbf{Início}       & inclusão e exclusão de disciplinas               & 14/08/23      \\
    2023.2            & \textbf{Fim}          & inclusão e exclusão de disciplinas               & 21/08/23      \\
    2023.2            & \textbf{Fim}          & período letivo                                   & 08/12/23      \\
    2023.2            & \textbf{\textit{Fim}} & entrega dos resultados à SECACAD                 & 15/12/23      \\

    % 2024.1          & \textbf{Início}       & renovação de matrícula                           & 26/02/24      \\
    % 2024.1          & \textbf{Fim}          & renovação de matrícula                           & 01/03/24      \\
    % 2024.1          & \textbf{Início}       & período letivo                                   & 04/03/24      \\
    \hline
  \end{tabular}
\end{table}

Levando em consideração as ações sugeridas na \autoref{subsec:burocracia-férias} e na \autoref{subsec:burocracia-troca}, um cronograma alterado seria o disposto na \autoref{tab:calendario_2023-Alterado}, onde as ações sugeridas foram aplicadas. Essas ações incluem a alteração do calendário acadêmico para que todo o primeiro semestre comece suas atividades uma semana antes. Além disso, define-se que o período final de criação de turmas não é efetivamente finalizado no antes da renovação de matrícula, mas sim após a finalização da inclusão e exclusão de disciplinas.

Nota-se que o único caso em que não foi feita a migração exata de uma semana foi na renovação de matrícula, visto que caso fosse alterada para uma semana exata, a renovação de matrícula seria feita no período de carnaval, o que não é desejado. Assim, foi mantido o mesmo intervalo de tempo, porém, começando após o feriado nacional ``Quarta-feira de Cinzas''.

\begin{table}[H] \centering \caption{Calendário Acadêmico de 2023 - Alterado} \label{tab:calendario_2023-Alterado}
  \begin{tabular}{| c r l r |}
    \hline
    \textbf{Vigência} & \textbf{Fase}         & \textbf{Atividades}                & \textbf{Data}                  \\
    \hline
    2023.1            & \textbf{\textit{Fim}} & cadastro de novas disciplinas      & \sout{20/01} 13/01/23 \altered \\
    2023.1            & \textbf{Início}       & criação de turmas                  & \sout{20/01} 13/01/23 \altered \\ \removeLine
    \sout{2023.1}     & \sout{\textbf{Fim}}   & \sout{criação de turmas}           & \sout{15/02/23}                \\
    2023.1            & \textbf{Início}       & renovação de matrícula             & \sout{28/02} 23/02/23 \altered \\
    2023.1            & \textbf{Fim}          & renovação de matrícula             & \sout{03/03} 26/02/23 \altered \\
    2023.1            & \textbf{Início}       & período letivo                     & \sout{06/03} 27/02/23 \altered \\
    2023.1            & \textbf{Início}       & inclusão e exclusão de disciplinas & \sout{06/03} 27/02/23 \altered \\
    2023.1            & \textbf{Fim}          & inclusão e exclusão de disciplinas & \sout{20/03} 13/03/23 \altered \\ \addLine
    2023.1            & \textbf{Fim}          & criação de turmas                  & 20/03/23                       \\
    2023.1            & \textbf{Fim}          & período letivo                     & \sout{07/07} 30/06/23 \altered \\
    2023.1            & \textbf{\textit{Fim}} & entrega dos resultados à SECACAD   & \sout{14/07} 07/07/23 \altered \\

    2023.2            & \textbf{\textit{Fim}} & cadastro de novas disciplinas      & 14/07/23                       \\
    2023.2            & \textbf{Início}       & criação de turmas                  & 17/07/23                       \\ \removeLine
    \sout{2023.2}     & \sout{\textbf{Fim}}   & \sout{criação de turmas}           & \sout{28/07/23}                \\
    2023.2            & \textbf{Início}       & renovação de matrícula             & 01/08/23                       \\
    2023.2            & \textbf{Fim}          & renovação de matrícula             & 04/08/23                       \\
    2023.2            & \textbf{Início}       & período letivo                     & 07/08/23                       \\
    2023.2            & \textbf{Início}       & inclusão e exclusão de disciplinas & 14/08/23                       \\
    2023.2            & \textbf{Fim}          & inclusão e exclusão de disciplinas & 21/08/23                       \\ \addLine
    2023.2            & \textbf{Fim}          & criação de turmas                  & 21/08/23                       \\
    2023.2            & \textbf{Fim}          & período letivo                     & 08/12/23                       \\
    2023.2            & \textbf{\textit{Fim}} & entrega dos resultados à SECACAD   & 15/12/23                       \\
    \hline
  \end{tabular}
\end{table}

Com a utilização desses dois métodos burocráticos, aqueles encarregados de realizar a criação e alocação das turmas nos devido horários disporá de uma semana extra no segmento entre os semestres, assim abrindo uma janela de tempo maior para a manipulação dos horários, e consequentemente, ampliando possibilidade de se alcançar uma solução ótima.

\section{Trabalhos futuros} % ## 8.1 Trabalhos futuros

% EVERYTHING IS A DRAFT

Como trabalhos futuros, vê-se uma ampla gama de pesquisa e aprimoramento ao presente trabalho, visto que este busca um método alternativo de solução ao mesmo problema abordado por outros dois pesquisadores em tempos anteriores. Pode-se então elaborar uma conexão entre o atual sistema e modelos aos métodos heurísticos propostos, permitindo então uma abordagem híbrida humano-computador na busca da grade horária ótima. Sugere-se inclusive o estudo sobre a aplicação de métodos de programação inteira, visto que através da revisão bibliográfica este método apresentou consideráveis resultados.

Assim como os modelos anteriores apresentaram diversas incongruências com a realidade prática da universidade estudada, é esperado que este trabalho acabe por trilhar o mesmo caminho, visto que o problema em questão realmente apresenta grande parte de sua complexidade no entendimento e modelagem de como as diversas partes da instituição interagem entre si, porém, espera-se que este documento possa servir como uma boa base para o entendimento de sua estrutura.

Quanto ao software, mesmo que o prioritário seja a sua funcionalidade, é esperado que o seu design seja o mais intuitivo, fluido e prático quanto for possível. Sendo esta tarefa direcionada mais à experiência do usuário, possivelmente tangenciando o problema central de construção de grades horárias.

Considerando que as duas tentativas anteriores resultaram em métodos que embora atingissem seu objetivo, não foram implementados na prática, tem-se como esperado que o mesmo ocorra com este trabalho. Com isso, espera-se que em trabalhos futuros se estude e analise os motivos de falha do uso prático do atual sistema.

% O TEU TRABALHO É APENAS UM PASSO, NÃO O FIM. Não queira fazer tudo, foque em fazer o máximo que você puder para que o trabalho possa ser melhor desenvolvido por outros.

\subsection{Aprimorando o experimento}

\begin{enumerate}
  \item \textbf{Preparação dos dados}: organizar a base de dados das informações anteriores para facilitar a criação da grade horária do semestre seguinte;
  \item \textbf{Preparação do ambiente de trabalho}: organização do ambiente de trabalho, com a separação dos materiais necessários para a realização do experimento;
  \item \textbf{Preenchimento da grade horária do CCT}: alocação da grade horária do CCT para o semestre seguinte;
  \item \textbf{Preenchimento da grade horária de Ciência da Computação}: criação manual da grade horária do curso de Ciência da Computação para o semestre seguinte;
  \item \textbf{Resolução de conflitos}: resolução de conflitos entre as grades horárias.
\end{enumerate}

Aumento de conflitos a serem resolvidos

\subsection*{Apelo}

Eu gostaria de deixar aqui um alerta para quem for utilizar este documento como base para futuros trabalhos: a maior dificuldade a ser superada é o fator organizacional. A minha percepção é de que a UENF atualmente se encontra tal qual um osso quebrado que se regenerou sem o uso de gesso para o fixar no local certo: funciona, mas não tão bem quanto seria capaz. E, assim como no caso ósseo, para que você atinja um resultado ótimo, certamente terá que quebrar algumas estruturas já consolidadas para que possa reorganizá-las de forma mais eficiente.

Neste trabalho tentei pavimentar o caminho na direção que acredito ser a mais apropriada para a adoção do sistema. Nesse caminho, acabei abrindo mão de meus desejos pessoais que envolviam o sistema direcionado às demandas dos alunos, visto que, mesmo que atingisse um resultado ótimo aos alunos, nada adiantaria se o sistema não fosse adequado àqueles que o usarão. Eu espero que este trabalho não se torne apenas mais uma monografia que será esquecida em uma prateleira, mas sim que ele possa ser utilizado como um guia para a construção do sistema que um dia sonhei em desenvolver.

Se você chegou até aqui, eu agradeço por ter lido este trabalho. E, se você for um estudante da UENF, eu peço que você não desista de lutar por um ensino melhor. A UENF é uma instituição que tem um grande potencial, e eu acredito que ela pode ser muito mais do que é hoje. Eu espero que este trabalho possa ser um pequeno passo na direção de um futuro melhor para a nossa universidade.

Caso o sistema ainda esteja em funcionamento, excelente, isso significa que consegui atingir um de meus objetivos, então, continue a aprimorá-lo. Caso contrário, torne como seu objetivo consertar os meus erros. Descubra o motivo da não adoção do sistema e corrija-o. E, se possível, me avise, eu adoraria saber que o meu trabalho não foi em vão.

Além do desenvolvimento da monografia como objetivo para a conclusão do curso, o que desejo é conseguir auxiliar as pessoas em suas atividades diárias. Ainda mais se considerarmos que este sistema, se bem executado, tende a ajudar semestralmente centenas, senão milhares, de alunos e professores semestralmente.
