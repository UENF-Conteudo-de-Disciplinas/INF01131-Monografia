% !TEX root = ./base.tex
%% abtex2-modelo-trabalho-academico.tex, v-1.9.5 laurocesar
%% Copyright 2012-2015 by abnTeX2 group at http://www.abntex.net.br/ 
%%
%% This work may be distributed and/or modified under the
%% conditions of the LaTeX Project Public License, either version 1.3
%% of this license or (at your option) any later version.
%% The latest version of this license is in
%%   http://www.latex-project.org/lppl.txt
%% and version 1.3 or later is part of all distributions of LaTeX
%% version 2005/12/01 or later.
%%
%% This work has the LPPL maintenance status `maintained'.
%% 
%% The Current Maintainer of this work is the abnTeX2 team, led
%% by Lauro César Araujo. Further information are available on 
%% http://www.abntex.net.br/
%%
%% This work consists of the files abntex2-modelo-trabalho-academico.tex,
%% abntex2-modelo-include-comandos and abntex2-modelo-references.bib
%%

% ------------------------------------------------------------------------
% ------------------------------------------------------------------------
% abnTeX2: Modelo de Trabalho Academico (tese de doutorado, dissertacao de
% mestrado e trabalhos monograficos em geral) em conformidade com 
% ABNT NBR 14724:2011: Informacao e documentacao - Trabalhos academicos -
% Apresentacao
% ------------------------------------------------------------------------
% ------------------------------------------------------------------------

\documentclass[
  % -- opções da classe memoir --
  12pt,        % tamanho da fonte
  openright,      % capítulos começam em pág ímpar (insere página vazia caso preciso)
  oneside,      % para impressão apenas no verso. Oposto a twoside
  % twoside,      % para impressão em verso e anverso. Oposto a oneside
  a4paper,      % tamanho do papel. 
  % -- opções da classe abntex2 --
  %chapter=TITLE,    % títulos de capítulos convertidos em letras maiúsculas
  %section=TITLE,    % títulos de seções convertidos em letras maiúsculas
  %subsection=TITLE,  % títulos de subseções convertidos em letras maiúsculas
  %subsubsection=TITLE,% títulos de subsubseções convertidos em letras maiúsculas
  % -- opções do pacote babel --
  english,      % idioma adicional para hifenização
  french,        % idioma adicional para hifenização
  spanish,      % idioma adicional para hifenização
  brazil        % o último idioma é o principal do documento
]{abntex2}

% ---
% Pacotes básicos 
% ---
\usepackage{lmodern}      % Usa a fonte Latin Modern      
\usepackage[T1]{fontenc}    % Selecao de codigos de fonte.
\usepackage[utf8]{inputenc}    % Codificacao do documento (conversão automática dos acentos)
\usepackage{lastpage}      % Usado pela Ficha catalográfica
\usepackage{indentfirst}    % Indenta o primeiro parágrafo de cada seção.
\usepackage{color}        % Controle das cores
\usepackage{graphicx}      % Inclusão de gráficos
\usepackage{microtype}       % para melhorias de justificação
\usepackage{booktabs}
\usepackage{graphicx}
\usepackage[table,xcdraw]{xcolor}
\usepackage{float}
% ---

% ---
% Pacotes adicionais, usados apenas no âmbito do Modelo Canônico do abnteX2
% ---
\usepackage{lipsum}        % para geração de dummy text
% ---

% --- Pacotes de citações ---

\usepackage[brazilian,hyperpageref]{backref}   % Paginas com as citações na bibl
\usepackage[alf]{abntex2cite}  % Citações padrão ABNT

% --- 
% CONFIGURAÇÕES DE PACOTES
% --- 

% ---
% Configurações do pacote backref
% Usado sem a opção hyperpageref de backref
\renewcommand{\backrefpagesname}{Citado na(s) página(s):~} % Texto padrão antes do número das páginas

\renewcommand{\backref}{}
% Define os textos da citação
\renewcommand*{\backrefalt}[4]{
  \ifcase#1 %
    Nenhuma citação no texto.%
  \or{}
    Citado na página #2.%
  \else
    Citado #1 vezes nas páginas #2.%
  \fi}%
% ---
\setlength{\parindent}{1.3cm} % --- Espaçamentos entre linhas e parágrafos --- % 
\setlength{\parskip}{0.2cm}  % Controle do espaçamento entre um parágrafo e outro: tente também \onelineskip
% Seleciona o idioma do documento (conforme pacotes do babel)
%\selectlanguage{english}
\selectlanguage{brazil}
\frenchspacing % Retira espaço extra obsoleto entre as frases.
\makeindex % --- compila o indice --- %


% ADIÇÕES FEITAS POR MIM (João Vítor Fernandes Dias)

% \usepackage{fontspec} % Adicionar emoji
% \usepackage{parskip} % Adicionar emoji?
\usepackage{booktabs} % Necessários pra tabela
\usepackage{multirow} % Necessários pra tabela
\usepackage{listings} % Necessário para código

% Necessário para códigos bonitos

\definecolor{codegreen}{HTML}{009900}
\definecolor{codegray}{HTML}{808080}
\definecolor{codepurple}{HTML}{9400D3}
\definecolor{backcolour}{HTML}{F2F2EB}

\lstdefinestyle{mystyle}{
  backgroundcolor=\color{backcolour},
  commentstyle=\color{codegreen},
  keywordstyle=\color{magenta},
  numberstyle=\tiny\color{codegray},
  stringstyle=\color{codepurple},
  basicstyle=\ttfamily\footnotesize,
  breakatwhitespace=false,
  breaklines=true,
  captionpos=b,
  keepspaces=true,
  numbers=left,
  numbersep=5pt,
  showspaces=false,
  showstringspaces=false,
  showtabs=false,
  tabsize=2
}

\lstset{style=mystyle}
