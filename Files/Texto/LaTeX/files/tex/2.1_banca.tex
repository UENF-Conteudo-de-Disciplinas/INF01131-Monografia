\chapter*{Banca}

% Por enquanto, a banca que eu tenho em mente é composta por:

% \begin{enumerate}
%   \item Fermín Alfredo Tang Montané (Orientador)
%   \item Annabell del Real Tamariz (Ex Chefe do Laboratório de Matemática)
%   \item Oscar Alfredo Paz La Torre (Ex Diretor do CCT)
%   \item Luiz Humberto Guillermo Felipe (Chefe do Laboratório de Matemática)
%   \item Diretora CCT (Diretora do CCT)
% \end{enumerate}

% Conforme resolução 004/2007 do COLAC, artigo 9 e parágrafo 1, a banca examinadora deverá ter a seguinte composição:
% (i) o Professor Orientador e/ou Co-orientador do aluno, que presidirá os trabalhos,
% (ii) um membro indicado, de comum acordo, pelo estudante e seu Professor Orientador ou Co-Orientador e
% (iii) um membro indicado pelo Colegiado do Curso. Em caráter excepcional, um dos três avaliadores poderá ser um Mestre ou doutorando ou pós doutorando que tenha formação compatível com o tema da monografia. Além dos membros titulares, deverá ser indicado um membro suplente. A composição da banca deverá ser aprovada pelo Colegiado do Curso, dando preferência para que o presidente seja doutor. Quando o orientador ou co-orientador estiver impossibilitado de estar presente na banca examinadora, o coordenador do Curso poderá representá-lo, desde que seja requerido por escrito e antecipadamente pelo orientador do aluno.
