
\begin{apendicesenv}
  \chapter{Formulário de pesquisa quantitativa}

  \begin{Form}[action=mailto:joaovitorfd2000@gmai.com, encoding=html, method=post]

    Pesquisa quantitativa de alunos da UENF sobre distribuição e oferta de disciplinas

    \section*{Pesquisa quantitativa de alunos da UENF sobre distribuição e oferta de disciplinas}

    Olá! Desde já agradeço por ceder em torno de 4 minutos do seu tempo para responder a este formulário usando o seu e-mail institucional. Considerando que nosso tempo é valioso, vamos direto ao objetivo:

    Me chamo João Vítor Fernandes Dias, estudante de Ciência da Computação na UENF, e estou fazendo minha Monografia. Ela trata da elaboração de um sistema para a coordenação de curso poder analisar mais facilmente quais são as disciplinas que serão disponibilizadas a cada semestre e a quais salas e professores serão atribuídas.

    O objetivo da minha monografia é conseguir tornar mais eficiente a distribuição das disciplinas, para que se resulte em um conjunto de disciplinas ofertadas com melhor qualidade. Espera-se com isso que as demandas de disciplina dos alunos sejam melhor atendidas, assim como as preferências de horários dos professores.

    Este formulário tem como objetivo avaliar a sua satisfação em relação ao processo de inscrição semestral nas disciplinas.

    \section*{Sobre você}

    Nesta seção, peço que informe algumas características suas para que a análise estatística se torne mais rica.

    \ChoiceMenu[print, combo]{Qual o seu curso?}
    {
      1. Administração Pública,
      2. Agronomia,
      3. Biologia (Licenciatura),
      4. Ciência da Computação,
      5. Ciências Biológicas (bacharelado),
      6. Ciências Sociais,
      7. Engenharia Civil,
      8. Engenharia de Exploração e Produção de Petróleo,
      9. Engenharia de Produção,
      10. Engenharia Metalúrgica,
      11. Engenharia Meteorológica,
      12. Física (licenciatura),
      13. Matemática (Licenciatura),
      14. Medicina Veterinária,
      15. Pedagogia (Licenciatura),
      16. Química (Licenciatura),
      17. Zootecnia,
      18. Outro,
    }

    \ChoiceMenu[print, combo]{Em que ano você ingressou na UENF?}
    {
      2023,
      2022,
      2021,
      2020,
      2019,
      2018,
      2017,
      2016,
      2015,
      2014,
      2013,
      Outro,
    }

    \section*{Pesquisa de satisfação}

    Agora serão feitas algumas perguntas em relação à sua satisfação com algumas características da Universidade.

    Abaixo, estão algumas perguntas gerais em relação à sua satisfação com a distribuição de disciplinas semestralmente.

    \begin{enumerate}
      \item \ChoiceMenu[print, combo]{\textbf{Salas}: Você já teve que mudar de sala por falta de algum acessório como quadro, projetor ou monitor?}
            {Sim, Não, Outro}
      \item \ChoiceMenu[print, combo]{\textbf{Salas}: Você já teve aula cuja sala não dispunha de carteiras o suficiente?}
            {Sim, Não, Outro}
      \item \ChoiceMenu[print, combo]{\textbf{Vagas}: Você já quis entrar em uma disciplina, mas ela não tinha vaga?}
            {Sim, Não, Outro}
      \item \ChoiceMenu[print, combo]{\textbf{Vagas}: Você já ficou acordado após meia-noite por medo de não ter vaga para as disciplinas que deseja cursar?}
            {Sim, Não, Outro}
      \item \ChoiceMenu[print, combo]{\textbf{Conflitos}: Você já deixou de se inscrever em uma disciplina por causa de conflito de horário?}
            {Sim, Não, Outro}
      \item \ChoiceMenu[print, combo]{\textbf{Preferências}: Você já preferiu não se inscrever em uma disciplina para cursá-la em outro momento mais oportuno?}
            {Sim, Não, Outro}
      \item \ChoiceMenu[print, combo]{\textbf{Opiniões}: Você acha que a universidade deveria oferecer horários diferentes para as disciplinas mais demandadas para evitar conflitos com outras disciplinas?}
            {Sim, Não, Outro}
    \end{enumerate}

    \section*{Preferências pessoais}

    Esta seção visa saber um pouco mais sobre as suas preferências pessoais quanto a escolha das disciplinas ofertadas.






  \end{Form}

\end{apendicesenv}

\begin{apendicesenv} % Imprime uma página indicando o início dos apêndices

  \chapter{Formulário de pesquisa quantitativa}

  Pesquisa quantitativa de alunos da UENF sobre distribuição e oferta de disciplinas

  \section*{Pesquisa quantitativa de alunos da UENF sobre distribuição e oferta de disciplinas}

  Olá! Desde já agradeço por ceder em torno de 4 minutos do seu tempo para responder a este formulário usando o seu e-mail institucional. Considerando que nosso tempo é valioso, vamos direto ao objetivo:

  Me chamo João Vítor Fernandes Dias, estudante de Ciência da Computação na UENF, e estou fazendo minha Monografia. Ela trata da elaboração de um sistema para a coordenação de curso poder analisar mais facilmente quais são as disciplinas que serão disponibilizadas a cada semestre e a quais salas e professores serão atribuídas.

  O objetivo da minha monografia é conseguir tornar mais eficiente a distribuição das disciplinas, para que se resulte em um conjunto de disciplinas ofertadas com melhor qualidade. Espera-se com isso que as demandas de disciplina dos alunos sejam melhor atendidas, assim como as preferências de horários dos professores.

  Este formulário tem como objetivo avaliar a sua satisfação em relação ao processo de inscrição semestral nas disciplinas.

  \section*{Sobre você}

  Nesta seção, peço que informe algumas características suas para que a análise estatística se torne mais rica.

  \begin{itemize}
    \item \textbf{Pergunta:} Qual o seu curso?
    \item \textbf{Opções de resposta}
          \begin{enumerate}
            \item Administração Pública
            \item Agronomia
            \item Biologia (Licenciatura)
            \item Ciência da Computação
            \item Ciências Biológicas (bacharelado)
            \item Ciências Sociais
            \item Engenharia Civil
            \item Engenharia de Exploração e Produção de Petróleo
            \item Engenharia de Produção
            \item Engenharia Metalúrgica
            \item Engenharia Meteorológica
            \item Física (licenciatura)
            \item Matemática (Licenciatura)
            \item Medicina Veterinária
            \item Pedagogia (Licenciatura)
            \item Química (Licenciatura)
            \item Zootecnia
            \item Outro
          \end{enumerate}
  \end{itemize}

  \begin{itemize}
    \item \textbf{Pergunta:} Em que ano você ingressou na UENF?
    \item \textbf{Opções de resposta}
          \begin{enumerate}
            \item 2023
            \item 2022
            \item 2021
            \item 2020
            \item 2019
            \item 2018
            \item 2017
            \item 2016
            \item 2015
            \item 2014
            \item 2013
            \item Outro
          \end{enumerate}
  \end{itemize}

  \section*{Pesquisa de satisfação}

  Agora serão feitas algumas perguntas em relação à sua satisfação com algumas características da Universidade.

  Abaixo, estão algumas perguntas gerais em relação à sua satisfação com a distribuição de disciplinas semestralmente.

  \begin{itemize}
    \item \textbf{Perguntas}
          \begin{enumerate}
            \item Salas: Você já teve que mudar de sala por falta de algum acessório como quadro, projetor ou monitor?
            \item Salas: Você já teve aula cuja sala não dispunha de carteiras o suficiente?
            \item Vagas: Você já quis entrar em uma disciplina, mas ela não tinha vaga?
            \item Vagas: Você já ficou acordado após meia-noite por medo de não ter vaga para as disciplinas que deseja cursar?
            \item Conflitos: Você já deixou de se inscrever em uma disciplina por causa de conflito de horário?
            \item Preferências: Você já preferiu não se inscrever em uma disciplina para cursá-la em outro momento mais oportuno?
            \item Opiniões: Você acha que a universidade deveria oferecer horários diferentes para as disciplinas mais demandadas para evitar conflitos com outras disciplinas?
          \end{enumerate}
    \item \textbf{Opções de resposta}
          \begin{enumerate}
            \item Sim
            \item Não
            \item Outro
          \end{enumerate}
  \end{itemize}

  \section*{Preferências pessoais}

  Esta seção visa saber um pouco mais sobre as suas preferências pessoais quanto a escolha das disciplinas ofertadas.

  \begin{itemize}
    \item \textbf{Pergunta:} Você prefere disciplinas distribuídas ao longo da semana ou acumuladas em poucos dias?
    \item \textbf{Opções de resposta}
          \begin{enumerate}
            \item Distribuídas ao longo da semana
            \item $\sim$
            \item Não tenho preferência
            \item $\sim$
            \item Acumuladas em poucos dias
          \end{enumerate}
  \end{itemize}

  \begin{itemize}
    \item \textbf{Pergunta:} Você prefere disciplinas na parte da manhã ou na parte da tarde?
    \item \textbf{Opções de resposta}
          \begin{enumerate}
            \item na parte da manhã
            \item $\sim$
            \item Não tenho preferência
            \item $\sim$
            \item na parte da tarde
          \end{enumerate}
  \end{itemize}

  \begin{itemize}
    \item Como você lida com conflitos de horário entre as disciplinas que deseja cursar?
    \item \textbf{Opções de resposta} (Permite múltiplas escolhas)
          \begin{itemize}
            \item Escolho a mais difícil
            \item Escolho a mais fácil
            \item Escolho a que tem mais créditos
            \item Escolho a que prefiro
            \item Escolho a que ``prende'' mais matérias
            \item Escolho a disciplina mais concorrida
            \item Outro...
          \end{itemize}
  \end{itemize}

  \section*{Experiências passadas com atrasos e disciplinas}

  Aqui estão algumas perguntas relacionadas à divergência entre o período esperado de conclusão das disciplinas VS o período em que elas de fato são realizadas.

  \begin{itemize}
    \item \textbf{Pergunta:} Quanto tempo (em períodos) você já teve que esperar para fazer uma disciplina da sua grade?
    \item Descrição: Exemplo hipotético: estou no 6º período e estou desde o 4º período tentando me inscrever em uma disciplina, mas ela não foi oferecida ou não teve vaga, então tive que esperar 2 períodos.
    \item \textbf{Opções de resposta}: 0, 1, 2, 3, 4, 5, 6, 7, 8, 9, 10
  \end{itemize}

  \begin{itemize}
    \item \textbf{Pergunta:} Qual foi a quantidade máxima de períodos que você se distanciou de uma disciplina de determinado período?
    \item Descrição: Exemplo hipotético: estou no 6º período da faculdade, mas ainda estou cursando uma disciplina do 3º período, pois escolhi não fazer antes, ou ainda não obtive a aprovação, logo, me distanciei 3 períodos do esperado.
    \item \textbf{Opções de resposta}: 0, 1, 2, 3, 4, 5, 6, 7, 8, 9, 10
  \end{itemize}

  \section*{Você acha que a distribuição de disciplinas semestrais é...}

  \begin{itemize}
    \item Classificações
          \begin{itemize}
            \item Justa (feita de acordo a atender os desejos da maioria)
            \item Variada (bem diversa e abrange diversos interesses)
            \item Contínua (oferecida de forma a ter aulas sequenciais)
            \item Eficiente (bem sucedida em atender aos desejos dos alunos)
            \item Distribuída (bem espaçada ao longo da semana)
            \item Satisfatória (agradável aos meus desejos pessoais)
          \end{itemize}
    \item \textbf{Descrição:} Exemplo hipotético: estou no 6º período da faculdade, mas ainda estou cursando uma disciplina do 3º período, pois escolhi não fazer antes, ou ainda não obtive a aprovação, logo, me distanciei 3 períodos do esperado.
          \begin{enumerate}
            \item Discordo completamente
            \item $\sim$
            \item $\sim$
            \item $\sim$
            \item Concordo completamente
          \end{enumerate}
  \end{itemize}

  \section*{Opcional}

  Por fim, deixo aqui um espaço caso deseje compartilhar algum comentário, opinião ou sugestão quanto ao meu trabalho ou formulário.

  Escreva aqui caso haja algo que gostaria de comentar, opinar ou sugerir. Tudo bem deixar em branco, suas informações já foram de grande ajuda.

  \begin{itemize}
    \item Campo de texto livre
  \end{itemize}

\end{apendicesenv}
