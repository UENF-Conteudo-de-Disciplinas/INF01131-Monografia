\chapter{Resultados} \label{chap:resultados} % ### 7.

Estando no início do fim, seguiremos agora numa jornada retroativa ao que foi feito até então e os resultados obtidos.

\section{Contexto Acadêmico} % ### 7.1.

Pudemos ver através da pesquisa acadêmica aos artigos e trabalhos relacionados com a área da construção de grades horárias em específico, o \textit{university course timetabling}, que a área é vasta, tanto em quantidade de artigos publicados quanto com muitas possibilidades de pesquisa e desenvolvimento que pode ser visto na revisão literária realizada por \citeonline{Alencar2019}. Uma das principais questões que mantêm a área em constante inconclusão é o nível de especificidade que cada instituição de ensino possui, o desafio deixa de ser a implementação do método resolutivo, e passa a ser modelar o problema específico e lidar com as preocupações dos usuários, como é concluído por \citeonline{Murray2007}.

A UENF não é diferente, já tendo sido alvo de pesquisas e desenvolvimentos de sistemas de otimização de grade horária em anos anteriores, em especial as monografias de \citeonline{Sanya2013} e \citeonline{Ricardo2014}, mas que não foram adotados pela instituição mesmo que representassem a eficiência da resolução do problema.

Uma alternativa encontrada para contornar a dificuldade encontrada pelos trabalhos anteriores na UENF no campo da modelagem do problema foi utilizar de uma abordagem voltada para a Interação Humano-Computador, como feito por \citeonline{Andre2018}, para a construção de um sistema de otimização de grade horária, que permitisse a participação ativa dos usuários na construção da grade horária, o que poderia facilitar a aceitação do sistema pela instituição.

\section{Estrutura da Instituição}

Visando enfrentar diretamente o problema da especificidade e modelagem do problema na UENF. Para tanto foi feito um estudo sobre a forma como os diversos setores relacionados à construção da grade horária interagem entre si, quais são os seus responsáveis e qual é a sequência de ações que cada setor realiza para a construção da grade horária.

O primeiro passo foi a leitura dos documentos oficiais da UENF, como o Estatuto \cite{Estatuto2002}, Regimento \cite{Regimento2012}, Normas da Graduação \cite{Normas2012} e o Projeto Pedagógico do Curso de Ciência da Computação \cite{PPCCC2015}.

% Não cito sobre todos eles na parte da organização, devo citar.

Com base nos documentos oficiais da UENF, foi possível identificar os setores responsáveis pela construção da grade horária, que são a Secretaria Acadêmica, a Direção de Centro, a Chefia do Laboratório e a Coordenação do Curso. Também relacionado com o processo de oferecimento das turmas está o Sistema Acadêmico da UENF, não sendo ele recorrentemente citado nas documentações, mesmo que no presente momento esteja interligado entre esses setores.

As atribuições de cada setor estão, em sua maioria, descritas nas documentações encontradas. De forma resumida, as atribuições de cada setor são:

\begin{itemize}
  \item \textbf{Secretaria Acadêmica}: elaborar e divulgar o Calendário Acadêmico; otimizar os recursos humanos; ampliar a oferta de disciplinas;
  \item \textbf{Câmara de Graduação}: aprovar e modificar o calendário acadêmico; sugerir vagas de bolsistas;
  \item \textbf{Direção de Centro}: designação semestral de professores responsáveis pelas disciplinas, após ouvir os Laboratórios, Colegiados e Coordenações;
  \item \textbf{Chefia do Laboratório}: atribuir carga horária didática aos docentes do laboratório e aos bolsistas; designar docente responsável por disciplina ofertada;
  \item \textbf{Coordenação do Curso}: coordenar a distribuição de estudantes do curso; indicar à chefia de laboratório as disciplinas a serem ofertadas ao curso coordenado;
  \item \textbf{Sistema Acadêmico}: articula parcialmente as informações entre os setores, como a oferta de disciplinas e a distribuição de estudantes.
\end{itemize}

% Tá faltando comentar mais sobre as atribuições de cada setor lá na parte de estrutura.

\subsection{Entrevistas}

Como o sistema pretendido é voltado para se enquadrar no contexto prático, viu-se necessário a realização de pesquisas qualitativas em forma de entrevista com os responsáveis de cada setor para entender suas percepções pessoais à realidade prática recorrente na instituição. Com isto, pode-se obter informações mais detalhadas.

Por parte da \textbf{Direção do CCT}, viu-se a existência de termos específicos para certos agrupamentos de disciplinas/turmas (anuais, ímpares, pares, ``de serviço'', ``Ciclo básico'', repetentes); a preferência por alocar disciplinas ofertadas a múltiplos cursos em horários fixos; a não preferência por horários casados (em um mesmo horário em dias de semana com um dia de distância entre si); recorrência majoritária de horários de turmas que duram 2 horas, mas regular necessidade de turmas com horários que durem 3 horas, sendo esses alocados preferencialmente às 10h da manhã; as alterações de alocação ocorrem até o final do período; alocar primeiro as disciplinas de serviço e que têm maior demanda encaminha os conflitos futuros a turmas menores de um mesmo curso, o que pode facilitar a resolução do conflito.

Com o \textbf{desenvolvedor do Sistema Acadêmico}

\section{Sistema}

O código fonte para o sistema desenvolvido está disponvível no \autoref{apendice:CodigoFonte}.

\subsection{Solução Ótima} % ### 7.2.

\subsection{Preparo para trabalhos futuros}

Ao longo de todo o desenvolvimento

\begin{MyCenteredFigure}
  \caption{Banco de Dados Final}
  \label{fig:BD_Final}
  % \includegraphics[width=\textwidth]{}
\end{MyCenteredFigure}

