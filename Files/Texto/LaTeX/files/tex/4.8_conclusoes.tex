\chapter{Conclusões} \label{chap:conclusoes}

% CONCLUIR CAPÍTULO POR CAPÍTULO

O problema de organização de grade horária no ensino superior tem sido amplamente estudado por diversos pesquisadores. Devido sua natureza multidimensional e com forte tendência a especificidades, este campo de estudo se mostra como amplo e complexo.

Através da revisão bibliográfica, foi possível observar que a maioria dos trabalhos se foca em um método heurístico de solução, onde se busca uma solução ótima, ou próxima do ótimo, através de um método de busca. Entretanto, o presente trabalho se propõe a uma abordagem diferente, onde se busca uma solução boa o suficiente para que seja utilizada na prática, mesmo que não seja ótima, isso através do método de manipulação manual dos dados.

Para este fim foi desenvolvido um sistema de suporte à decisão para auxiliar os setores da Universidade Estadual do Norte Fluminense Darcy Ribeiro (UENF) responsáveis pela criação de grades horárias. O sistema foi desenvolvido com o intuito de ser utilizado como uma ferramenta auxiliar, onde os usuários possam manipular os dados de forma mais intuitiva e visual, assim reduzindo a necessidade de retrabalho e aumentando a produtividade.

O sistema permite que as quatro operações básicas de armazenamento persistente, sendo elas a criação, leitura, edição e deleção de dados. Com isso, os usuários podem adicionar manualmente as informações referentes ao trabalho de criação de grade horária de forma centralizada, assim reduzindo a necessidade de se lidar com diversos arquivos e planilhas. Facilitando também a visualização de informações, como a alocação de turmas, que pode ser visualizada de forma gráfica, assim facilitando a identificação de conflitos e problemas. O que consequentemente tende a agilizar o processo de busca por novas soluções e a redução dos conflitos.

% Tendo atingido este objetivo, espera-se que o sistema possa ser utilizado na prática, e que possa ser aprimorado e melhorado com o tempo.

\section{Trabalhos futuros} % ## 8.1 Trabalhos futuros

% EVERYTHING IS A DRAFT

Como trabalhos futuros, vê-se uma ampla gama de pesquisa e aprimoramento ao presente trabalho, visto que este busca um método alternativo de solução ao mesmo problema abordado por outros dois pesquisadores em tempos anteriores. Pode-se então elaborar uma conexão entre o atual sistema e modelos aos métodos heurísticos propostos, permitindo então uma abordagem híbrida humano-computador na busca da grade horária ótima. Sugere-se inclusive o estudo sobre a aplicação de métodos de programação inteira, visto que através da revisão bibliográfica este método apresentou consideráveis resultados.

Assim como os modelos anteriores apresentaram diversas incongruências com a realidade prática da universidade estudada, é esperado que este trabalho acabe por trilhar o mesmo caminho, visto que o problema em questão realmente apresenta grande parte de sua complexidade no entendimento e modelagem de como as diversas partes da instituição interagem entre si, porém, espera-se que este documento possa servir como uma boa base para o entendimento de sua estrutura.

Quanto ao software, mesmo que o prioritário seja a sua funcionalidade, é esperado que o seu design seja o mais intuitivo, fluido e prático quanto for possível. Sendo esta tarefa direcionada mais à experiência do usuário, possivelmente tangenciando o problema central de construção de grades horárias.

Considerando que as duas tentativas anteriores resultaram em métodos que embora atingissem seu objetivo, não foram implementados na prática, tem-se como esperado que o mesmo ocorra com este trabalho. Com isso, espera-se que em trabalhos futuros se estude e analise os motivos de falha do uso prático do atual sistema.

% O TEU TRABALHO É APENAS UM PASSO, NÃO O FIM. Não queira fazer tudo, foque em fazer o máximo que você puder para que o trabalho possa ser melhor desenvolvido por outros.

\subsection{Aprimorando o experimento}

\begin{enumerate}
  \item \textbf{Preparação dos dados}: organizar a base de dados das informações anteriores para facilitar a criação da grade horária do semestre seguinte;
  \item \textbf{Preparação do ambiente de trabalho}: organização do ambiente de trabalho, com a separação dos materiais necessários para a realização do experimento;
  \item \textbf{Preenchimento da grade horária do CCT}: alocação da grade horária do CCT para o semestre seguinte;
  \item \textbf{Preenchimento da grade horária de Ciência da Computação}: criação manual da grade horária do curso de Ciência da Computação para o semestre seguinte;
  \item \textbf{Resolução de conflitos}: resolução de conflitos entre as grades horárias.
\end{enumerate}

Aumento de conflitos a serem resolvidos

\subsection*{Apelo}

Eu gostaria de deixar aqui um alerta para quem for utilizar este documento como base para futuros trabalhos: a maior dificuldade a ser superada é o fator organizacional. A minha percepção é de que a UENF atualmente se encontra tal qual um osso quebrado que se regenerou sem o uso de gesso para o fixar no local certo: funciona, mas não tão bem quanto seria capaz. E, assim como no caso ósseo, para que você atinja um resultado ótimo, certamente terá que quebrar algumas estruturas já consolidadas para que possa reorganizá-las de forma mais eficiente.

Neste trabalho tentei pavimentar o caminho na direção que acredito ser a mais apropriada para a adoção do sistema. Nesse caminho, acabei abrindo mão de meus desejos pessoais que envolviam o sistema direcionado às demandas dos alunos, visto que, mesmo que atingisse um resultado ótimo aos alunos, nada adiantaria se o sistema não fosse adequado àqueles que o usarão. Eu espero que este trabalho não se torne apenas mais uma monografia que será esquecida em uma prateleira, mas sim que ele possa ser utilizado como um guia para a construção do sistema que um dia sonhei em desenvolver.

Se você chegou até aqui, eu agradeço por ter lido este trabalho. E, se você for um estudante da UENF, eu peço que você não desista de lutar por um ensino melhor. A UENF é uma instituição que tem um grande potencial, e eu acredito que ela pode ser muito mais do que é hoje. Eu espero que este trabalho possa ser um pequeno passo na direção de um futuro melhor para a nossa universidade.

Caso o sistema ainda esteja em funcionamento, excelente, isso significa que consegui atingir um de meus objetivos, então, continue a aprimorá-lo. Caso contrário, torne como seu objetivo consertar os meus erros. Descubra o motivo da não adoção do sistema e corrija-o. E, se possível, me avise, eu adoraria saber que o meu trabalho não foi em vão.

Além do desenvolvimento da monografia como objetivo para a conclusão do curso, o que desejo é conseguir auxiliar as pessoas em suas atividades diárias. Ainda mais se considerarmos que este sistema, se bem executado, tende a ajudar semestralmente centenas, senão milhares, de alunos e professores semestralmente.
