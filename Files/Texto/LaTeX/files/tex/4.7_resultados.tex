\chapter{Resultados} % ## 7. Resultados

Como esperado, encontrar uma solução ótima para o problema de criação de grade horária tende à impraticabilidade, visto a dificuldade de se definir se de fato há tal solução a ser atingida. Mesmo assim, ao se utilizar de um método de manipulação manual dos dados, foi possível obter uma solução que se aproxima da ótima, e que pode ser utilizada na prática.

Em termos quantitativos, o sistema desenvolvido apresenta uma redução de conflitos em relação à solução inicial. O que mostra um resultado satisfatório.

% Em termos quantitativos, o sistema desenvolvido apresenta uma redução de XX\% de conflitos em relação à solução inicial. O que mostra um resultado satisfatório.

Esse resultado, entretanto, não diz respeito à uma conclusão absoluta, visto que o sistema desenvolvido não foi testado em um ambiente real, e sim em um ambiente hipotético. Com isso, considerável parte de informações encontra-se faltante e foi substituída por dados aleatórios, o que pode ter influenciado no resultado final.

\section{Soluções Burocráticas} % ### 7.1. Soluções Burocráticas

Além da busca pela solução ótima, o presente trabalho também se propõe a buscar métodos ainda mais alternativos para se amenizar a problemática abordada. Sendo, de forma simples, o uso de meios burocráticos disponíveis na instituição que abre alguns caminhos para a solução do problema. Entretanto, é necessário que se tenha em mente que a burocracia é um processo lento e que pode ser desgastante, sendo até mesmo esperado que não seja desejado por parte dos construtores da grade horária.

\subsection{Tempo de elaboração das grades} % #### 7.1.1. Tempo de elaboração das grades

Durante as entrevistas, uma alternativa válida para a amenização da problemática abordada é a alteração do calendário anual da UENF que define férias de duas semanas entre os semestres. Caso seu calendário seja alterado para que as férias sejam de duas semanas, o problema de agendamento teria maior tempo para ser resolvido, assim fazendo com que a solução ótima seja provável de ser alcançada.

% Citar o Estatuto da forma correta.

Segundo o Artigo 28 do Estatuto da UENF, compete à secretaria acadêmica a elaboração da proposta de calendário escolar para que seja aprovado pelo Colegiado Acadêmico. Enquanto que o Artigo 63 da seção 2 do capítulo 1, informa que os calendários do curso de graduação devem ser aprovados pelas correspondentes câmaras, com observância do calendário da universidade.

Logo, quanto à alteração do calendário acadêmico, a alteração mostra-se como possível, sendo necessário apenas que o processo burocrático necessário seja enfrentado.

\subsection{Alteração forçada de horários} % #### 7.1.2. Alteração forçada de horários

Segundo o parágrafo primeiro do artigo 36 das Normas de Graduação, "qualquer alteração de horário/turno após o período de matrícula deverá ter a anuência por escrito de todos os discentes matriculados na turma". Seguindo ao segundo parágrafo do mesmo artigo, temos que "a alteração de horário das aulas da turma deverá ter a anuência da Coordenação de Curso e a ciência do Chefe do Laboratório responsável pela disciplina".

Sendo assim possível alterar os horários de aula caso seja necessária para que haja uma melhora geral na distribuição das turmas na grade horária, mais uma vez sendo necessário superar o processo burocrático necessário.
