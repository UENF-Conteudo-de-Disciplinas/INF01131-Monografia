% ----------------------------------------------------------

\chapter[Objetivos]{Objetivos}
\begin{enumerate}
  \item  Entender o cenário de Veículos Autônomos no mundo, e contrastar com o brasileiro:
        \begin{enumerate}
          \item Compreender o cenário automobilístico brasileiro, e as suas expectativas para essa tecnologia.
          \item Contrastar o mercado de veículos autônomos mundial com o brasileiro, buscando  decifrar o que é necessário para a aplicação dessa tecnologia no país.
        \end{enumerate}
  \item  Estudar as principais empresas de pesquisa que trabalham com Veículos Autônomos no mundo, e o que buscam economicamente e tecnologicamente no setor:
        \begin{enumerate}
          \item Identificar se buscam diferentes tipos de Carros Autônomos. Assim como entender as suas possíveis principais diferenças.
          \item Entender o que essas empresas buscam alcançar economicamente, e tecnologicamente ao inserir essa tecnologia no mercado.
          \item Conhecer as mudanças econômicas que carros autônomos podem trazer para a sociedade brasileira.
        \end{enumerate}
  \item Mapear as tecnologias essenciais para a Direção Autônoma:
        \begin{enumerate}
          \item Documentar quais são os \textit{Softwares}, algoritmos de controle, e sensores usados nesses veículos.
        \end{enumerate}
\end{enumerate}

\chapter[Metodologia]{Metodologia}

Baseado no ``Project-based learning'' \cite{krajcik2006project}. Seguiremos os estudos através de um projeto que aborda problemas do mundo real, cujo muitos não tem resposta única. Ao longo desse projeto será possível fazer novas perguntas e encontrar suas possíveis respostas por meio de uma investigação sustentada.

\vspace {1mm}

Este Plano de Pesquisa também utilizará as seguintes metodologias:
\begin{itemize}
  \item \textit{Pesquisa Exploratória; visando promover o enriquecimento do conhecimento sobre os diferentes assuntos relacionados a IA, ML, e Veículos Autônomos:
        }
        \begin{itemize}
          \item \textit{Levantamento Bibliográfico;}
          \item \textit{Levantamento documental;}
          \item \textit{Seminários e minicursos;}
          \item \textit{Participação em eventos;}
          \item \textit{Obtenção de experiências.}
        \end{itemize}
\end{itemize}

\section{Veículos Autônomos no Brasil e no mundo}

Nesta fase, iremos entender e fazer um estudo bibliográfico das iniciativas e expectativas do Brasil e do mundo para essa tecnologia. Atrelado a isso, faremos uma análise para identificar quais são os fatores necessários para aplicação dessa tecnologia pelo o mundo, sobretudo, no Brasil. Durante essas pesquisas iremos bibliografar os achados assim formando uma mapa estruturado das principais pesquisas e trabalhos na área.

\section{Veículos Autônomos e suas perspectivas}

O entendimento das perspectivas sociais e econômicas de uma tecnologia é vital para que possamos alocar recursos, e gerar mão de obra qualificada para o desenvolvimento de Tecnologias Disruptivas \cite{4cenarios_ocidental}, cujo sao inovações que são responsáveis por trazer grandes mudanças para o mercado e impulsionar tecnologicamente a sociedade, assim atendendo necessidades futuras da sociedade.
Dessa forma,  analisaremos acervos, pesquisas, e projetos. A fim de identificar as tendências econômicas, tecnológicas e sociais dessa tecnologia.

\section{Tecnologias Essenciais para a Direção Autônoma}
Nesta etapa, iremos mapear as tecnologias usadas para o desenvolvimento de um veículo autônomo. Com o intuito de compreender os seus recursos fundamentais e funcionalidades nesse tipo de veículo.  Portanto, realizaremos um estudo bibliográfico com o intuito de entender os principais algoritmos, software e artifícios físicos (hardware) usados em Carros Autônomos. Dessa forma, se tornaram nítidos quais são os recursos e conhecimentos necessários para o desenvolvimento e implementação desses veículos no país. Neste ano, propomos como ponto de partida a leitura do livro \cite{aurelien2017hands}, buscando entender cada um dos modelos apresentados referentes ao campo de Veículos Autônomos.

\chapter{Objetivo em Etapas} \label{chap:etapas}
A fim de alcançar os objetivos do Projeto de Pesquisa, neste plano listamos as principais atividades do cronograma \textbf{\ref{atividades}} que serão realizadas durante o período e vigência deste projeto:

\begin{enumerate}
  \item  Levantamento Bibliográfico das perspectivas nacional e internacional no que diz respeito a veículos autônomos;
  \item  Levantamento documental para compreender o que busca economicamente e tecnologicamente o mercado internacional e nacional em relação a veículos autônomos;
  \item Pesquisar quais são os diferentes tipos de veículos autônomos;
  \item Pesquisa bibliográfica das tecnologias essenciais de um carro autônomo;
  \item Mapear e entender os principais \textit{softwares} de controle de um carro autônomo;
  \item Elaboração de Relatório com os levantamentos.
\end{enumerate}
