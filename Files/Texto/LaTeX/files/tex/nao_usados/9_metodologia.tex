\chapter{Metodologia}

Considerando as dificuldades encontradas em trabalhos anteriores, entende-se que o maior desafio será superar as especificidades que serão encontradas durante a modelagem da universidade em questão. Para isso, será inicialmente necessária uma pesquisa bibliográfica com foco no estudo das abordagens qualitativas realizadas anteriormente que obtiveram sucesso em elicitar os requisitos adequados para as instituições de ensino.

% Adicionar referência sobre pesquisa qualitativa?

Com este conhecimento, um material inicial para a pesquisa exploratória e qualitativa deve ser desenvolvido levando em conta as questões próprias da universidade em questão, visando também coletar dados relevantes para uma futura pesquisa com maior enfoque em características emergentes que a pesquisa anterior pode levantar.

Nesta primeira pesquisa, algumas informações esperadas revolvem em torno das percepções dos \textit{stakeholders} do sistema proposto, sendo esses principalmente os professores, coordenadores de cursos, chefes de laboratório e diretores de centro. Estas percepções incluem o entendimento deles quanto ao método atual e às alternativas existentes, nível de insatisfação com o método atual, nível de desejo quanto à um novo método, características e funcionalidades que gostariam de ter em um sistema de suporte à decisão. Solicitando também que deem informações adicionais que gostariam de acrescentar. Questionamentos similares também serão realizados com alunos, porém em formato de formulário online para facilitar o processamento dos dados coletados.

Tendo as informações dos \textit{stakeholders} primários em mãos, diagramas conceituais devem ser desenhados utilizando \textit{softwares} de suporte como o \href{https://www.visual-paradigm.com/}{Visual Paradigm}, \href{drawio.com}{Draw.io} e/ou a \href{https://mermaid.js.org/}{ferramenta Mermaid}.

Munido destes dados, um sistema de suporte à decisão poderá ser desenvolvido para então ser usado como ferramenta centralizada, assim fazendo com que o objetivo geral seja alcançado.
