% resumo em português
\setlength{\absparsep}{18pt} % ajusta o espaçamento dos parágrafos do resumo
\begin{resumo}

    Este artigo visa apresentar uma análise da situação em que atualmente se encontra a criação de grades horárias na Universidade Estadual do Norte Fluminense, bem como apontar seus maiores problemas e desenvolver um sistema de decisão capaz de auxiliar diversos centros e laboratórios no desenvolvimento de suas grades horárias, buscando a otimização do uso de salas e redução dos conflitos existentes entre demandas de diferentes alunos por matérias.
    
    \textbf{Palavras-chave}: Tabela de horários, Agendamento de aulas universitárias, Heurísticas, Programação Inteira, Representação do Conhecimento, Interação Homem Computador.

\end{resumo}
