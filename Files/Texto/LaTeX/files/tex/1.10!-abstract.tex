% resumo em inglês
\begin{resumo}[Abstract]
  \begin{otherlanguage*}{english}

    Esta monografia apresenta um estudo sobre a criação de grades horárias em universidades, com foco na Universidade Estadual do Norte Fluminense, e o desenvolvimento de um sistema de suporte à decisão para a criação de grades horárias. Este processo é complexo e envolve diversos fatores, como a disponibilidade de salas e professores, a quantidade de alunos e a demanda por disciplinas. Como uma das maiores dificuldades no ramo acadêmico da criação de grades horárias é a modelagem do problema, este trabalho se propõe a estruturar as informações referentes à instituição estudada através de pesquisas e entrevistas. A partir disso, foi possível identificar a sequência de criação das grades e as restrições impostas pela instituição, podendo assim propor um sistema que auxilie o processe. Para tanto, o sistema foi desenvolvido utilizando JavaScript e a biblioteca React, sendo hospedado no GitHub Pages e utilizando da Amazon Web Services para a execução de seu \textit{backend} e armazenamento de dados. Suas funcionalidades incluem as funcionalidades CRUD (criar, ler, atualizar e deletar) de professores, disciplinas, salas, alunos, turmas e horários, a criação automatizada de uma grade horária inicial para o curso de Ciência da Computação da UENF, a visualização de conflitos dentre os recursos alocados e a análise história dos dados armazenados no sistema.

    This monograph presents a study on the creation of timetables in universities, focusing on the Universidade Estadual do Norte Fluminense, and the development of a decision support system for the creation of timetables. This process is complex and involves several factors, such as the availability of classrooms and teachers, the number of students and the demand for disciplines. As one of the greatest difficulties in the academic field of creating timetables is the modeling of the problem, this work proposes to structure the information regarding the institution studied through research and interviews. From this, it was possible to identify the sequence of creating the timetables and the restrictions imposed by the institution, thus being able to propose a system that assists the process. For this, the system was developed using JavaScript and the React library, being hosted on GitHub Pages and using Amazon Web Services for the execution of its backend and data storage. Its features include the CRUD (create, read, update and delete) functionalities of teachers, disciplines, classrooms, students, classes and schedules, the automated creation of an initial timetable for the Computer Science course at UENF, the visualization of conflicts among the allocated resources and the historical analysis of the data stored in the system.

    \textbf{Keywords}: Timetabling. University Class Scheduling. Heuristics. Integer Programming. Knowledge Organization. Human Computer Interaction.

  \end{otherlanguage*}

  % \vspace{\onelineskip}
  % \noindent

\end{resumo}