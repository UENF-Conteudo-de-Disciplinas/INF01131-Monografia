% resumo em português
\setlength{\absparsep}{18pt} % ajusta o espaçamento dos parágrafos do resumo
\begin{resumo}

  Este artigo visa apresentar uma análise da situação em que atualmente se encontra a criação de grades horárias na Universidade Estadual do Norte Fluminense, bem como apontar seus maiores problemas e desenvolver um sistema de decisão capaz de auxiliar diversos centros e laboratórios no desenvolvimento de suas grades horárias, buscando a otimização do uso de salas e redução dos conflitos existentes entre demandas de diferentes alunos por matérias.

  \textbf{Palavras-chave}: tabela de horários. agendamento de aulas universitárias. heurísticas. representação do conhecimento. interação humano-computador.

  % Resumo

  % Segundo a ABNT (2003, 3.1-3.2), o resumo deve ressaltar o objetivo, o método, os resultados e as conclusões do documento. A ordem e a extensão destes itens dependem do tipo de resumo (informativo ou indicativo) e do tratamento que cada item recebe no documento original. O resumo deve ser precedido da referência do documento, com exceção do resumo inserido no próprio documento. (. . . ) As palavras-chave devem figurar logo abaixo do resumo, antecedidas da expressão Palavras-chave:, separadas entre si por ponto e finalizadas também por ponto.

  % Palavras-chave: latex. abntex. editoração de texto.

\end{resumo}
