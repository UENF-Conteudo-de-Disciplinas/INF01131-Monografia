% resumo em português
\setlength{\absparsep}{18pt} % ajusta o espaçamento dos parágrafos do resumo
\begin{resumo}

  Esta monografia apresenta um estudo sobre a criação de grades horárias em universidades, com foco na Universidade Estadual do Norte Fluminense, e o desenvolvimento de um sistema de suporte à decisão para a criação de grades horárias. Este processo é complexo e envolve diversos fatores, como a disponibilidade de salas e professores, a quantidade de alunos e a demanda por disciplinas. Como uma das maiores dificuldades no ramo acadêmico da criação de grades horárias é a modelagem do problema, este trabalho se propõe a estruturar as informações referentes à instituição estudada através de pesquisas e entrevistas. A partir disso, foi possível identificar a sequência de criação das grades e as restrições impostas pela instituição, podendo assim propor um sistema que auxilie o processe. Para tanto, o sistema foi desenvolvido utilizando JavaScript e a biblioteca React, sendo hospedado no GitHub Pages e utilizando da Amazon Web Services para a execução de seu \textit{backend} e armazenamento de dados. Suas funcionalidades incluem as funcionalidades CRUD (criar, ler, atualizar e deletar) de professores, disciplinas, salas, alunos, turmas e horários, a criação automatizada de uma grade horária inicial para o curso de Ciência da Computação da UENF, a visualização de conflitos dentre os recursos alocados e a análise história dos dados armazenados no sistema.

  \textbf{Palavras-chave}: Tabelamento de Horários. Agendamento de Aulas Universitárias. Heurísticas. Design de Interação. Interação Humano-computador.

  % Resumo

  % Segundo a ABNT (2003, 3.1-3.2), o resumo deve ressaltar o objetivo, o método, os resultados e as conclusões do documento. A ordem e a extensão destes itens dependem do tipo de resumo (informativo ou indicativo) e do tratamento que cada item recebe no documento original. O resumo deve ser precedido da referência do documento, com exceção do resumo inserido no próprio documento. (. . . ) As palavras-chave devem figurar logo abaixo do resumo, antecedidas da expressão Palavras-chave:, separadas entre si por ponto e finalizadas também por ponto.

  % Palavras-chave: latex. abntex. editoração de texto.

\end{resumo}
