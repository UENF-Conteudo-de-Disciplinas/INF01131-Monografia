\chapter{Experimento temporal} \label{chap:experimentos}

Como o sistema tem por fim o auxílio na criação de grades horárias, é necessário averiguar a sua eficiência e eficácia. Para isso, foi realizado um experimento com o software desenvolvido, utilizando dados reais e hipotéticos, a fim de validar a sua aplicabilidade e funcionalidade.

O experimento consiste em cronometrar o tempo gasto para a criação da grade horária do semestre seguinte, utilizando o sistema desenvolvido e comparando com o tempo gasto para a criação manual de uma grade horária. Para isso, deve-se utilizar as informações disponíveis sobre as alocações de turmas anteriores a fim de criar uma grade horária.

\section{Sequência das atividades}

Como forma de permitir futuras comparações e regular quais são as tarefas que serão metrificadas, foi definida uma sequência de atividades a serem realizadas durante o experimento. A sequência de atividades é a seguinte:

\subsection{Descrição das atividades}

Abaixo estão descritas com um pouco mais de detalhes sobre as atividades a serem realizadas durante o experimento e o seu escopo.

\begin{enumerate}
  \item \textbf{Preparação dos dados}: organizar a base de dados das informações anteriores para facilitar a criação da grade horária do semestre seguinte;
  \item \textbf{Preparação do ambiente de trabalho}: organização do ambiente de trabalho, com a separação dos materiais necessários para a realização do experimento;
  \item \textbf{Preenchimento da grade horária do CCT}: alocação da grade horária do CCT para o semestre seguinte;
  \item \textbf{Preenchimento da grade horária de Ciência da Computação (CC)}: criação da grade horária do curso de Ciência da Computação para o semestre seguinte;
  \item \textbf{Resolução de conflitos}: resolução de conflitos entre as grades horárias.
\end{enumerate}

\subsection{Limitações}

Embora o experimento vise representar o tempo gasto em um cenário próximo do real, diversas implicações impedem que esta métrica seja uma representação fiel da realidade. Dentre as limitações, destacam-se:
Embora este experimento vise ilustrar o funcionamento do sistema desenvolvido, deve-se ressaltar as aproximações realizadas e as limitações do experimento, visto que podem interferir na precisão e confiabilidade dos resultados obtidos. Dentre as limitações e ressalvas, destacam-se:

\begin{enumerate}
  \item \textbf{Iteratividade}: o processo de criação de uma grade horária é iterativo, sendo necessárias diversas mudanças durante o processo, o que não será refletido no experimento que medirá apenas um ciclo dessa iteratividade. Essa situação é refletida tanto durante a alocação inicial das turmas que têm informações incompletas, quanto nos resultantes conflitos que surgem com a então alocação;
  \item \textbf{Familiaridade com o sistema}: o desenvolvedor do sistema é quem realizará o experimento, o que interfere diretamente com o nível de familiaridade com o sistema, consequentemente influenciando também na velocidade de uso;
  \item \textbf{Preferências}: embora em teoria não haja preferência por horários e salas, a coordenação do curso, com o tempo, passa a perceber os padrões e preferências dos professores e alunos, o que pode influenciar na alocação das turmas, o que não será refletido no experimento;
  \item \textbf{Variabilidade dos dados semestrais}: os dados utilizados para a realização do experimento são referentes ao ano anterior, o que pode não refletir a realidade do semestre seguinte, visto que a disponibilidade de professores, salas e disciplinas pode variar de um semestre para o outro;
        \begin{enumerate}
          \item \textbf{Professores temporários}: serão considerados que os mesmos professores temporários disponíveis no ano anterior estarão também disponíveis no ano seguinte, mesmo que, considerando a
                \hyperref[chap:instituicao]{realidade da instituição},
                isso não seja uma verdade absoluta;
          \item \textbf{Dados estruturados}: os dados utilizados já estão estruturados e organizados, então, não será computado o tempo que se levaria para assimilar e organizar as diversas informações advindas de diversas fontes, reduzindo também as esperadas faltas de informações e inconsistências nos dados;
          \item \textbf{Demandas imprecisas}: cada turma criada para o curso de Ciência da Computação tem seu valor de demanda estimada, e que, como o próprio nome descreve, é uma estimativa. Em sua criação, o valor de demanda é definido através das médias de demandas anteriores para determinada disciplina, o que pode não ser uma representação fiel da realidade, visto que diversos outros fatores influem no valor da demanda. Como consequência, os conflitos gerados talvez não sejam os mesmos que seriam gerados com a demanda real, podendo ser tanto mais simples quanto mais complexos;
        \end{enumerate}
\end{enumerate}

\subsection{Expectativa de resultado}

\begin{enumerate}
  \item \textbf{Preparação dos dados}: essa etapa tende a ter um caráter mais pessoal, podendo até mesmo não haver necessidade, seja por já possuir os dados organizados ou não se desejar organizar prematuramente. O atual testador tende a ter os dados organizados, o que deve reduzir o tempo gasto nessa etapa, em contrapartida, também pode-se acabar perdendo tempo com detalhes desnecessários;
  \item \textbf{Preparação do ambiente de trabalho}: assim como a anterior, esta tende também a ser uma etapa mais pessoal, mas espera-se que não passe de 5 minutos para poder pegar um copo d'água, abrir os arquivos necessários, etc.;
  \item \textbf{Preenchimento da grade horária do CCT}: espera-se que esta etapa seja a mais demorada, visto que a alocação de turmas do CCT atualmente será feita se baseando nos dados estruturados do ano anteriro, não dispondo da possibilidade de se importar diretamente os dados ou de duplicar as alocações anteriores;
  \item \textbf{Preenchimento da grade horária de Ciência da Computação}: esta etapa deve ser uma das mais rápidas, visto que distribuirá as turmas automaticamente;
  \item \textbf{Resolução de conflitos}: embora não apresente um método resolutivo automatizado, espera-se que não sejam necessários muito mais do que 10 minutos para se resolver todos os problemas.
\end{enumerate}

\section{Realização do experimento}

O experimento foi realizado no dia 30 de abril de 2024, na casa do desenvolvedor, utilizando um computador pessoal de modelo \LinkToURL{\LinkNotebook}{X571GT-AL888T} com o sistema operacional Windows 11.

Para a metrificação do tempo utilizou-se o \textit{software} \LinkToURL{\LinkLiveSplit}{\textbf{LiveSplit}} para a cronometragem das atividades. Dessa forma, foi possível obter as marcas temporais dispostas na \autoref{table:exp-temporal}. A marcação consiste em manualmente marcar o início e o fim de cada atividade, obtendo assim o tempo gasto em cada uma.

Como preparação para o experimento, foram organizados os arquivos necessários para a realização do experimento, sendo eles o arquivo PDF com as informações das turmas do CCT\dots

\section{Resultados}

Seguindo a sequência de atividades proposta, foram obtidas as marcas temporais dispostas na \autoref{table:exp-temporal}.

\begin{table}[htbp]\centering
  \caption{Tabela de tempos}
  \label{table:exp-temporal}
  \begin{tabular}{| l c c |}
    \hline
    \textbf{Etapa}                        & \textbf{Duração acumulada} & \textbf{Duração} \\
    \hline
    Preparação dos dados                  & 02:50                      & 00:00            \\
    Preparação do ambiente de trabalho    & 00:00                      & 00:00            \\
    Preenchimento da grade horária do CCT & 00:00                      & 00:00            \\
    Preenchimento da grade horária de CC  & 00:00                      & 00:00            \\
    Resolução de conflitos                & 00:00                      & 00:00            \\
    \hline
  \end{tabular}
\end{table}

Ao final do preenchimento da grade horária de Ciência da Computação, como esperado, alguns conflitos foram gerados, que foram resolvidos em seguida. As suas categorias e quantidades são dispostas na \autoref{table:exp-conflitos}.

\begin{table}[htbp]\centering
  \caption{Conflitos restantes e resolvidos}
  \label{table:exp-conflitos}
  \begin{tabular}{| l l c c |}
    \hline
    \textbf{Entidade} & \textbf{Conflito} & \textbf{Quantidade inicial} & \textbf{Quantidade final} \\
    \hline
    Professor         & Aloção múltipla   & 0                           & 0                         \\
    Sala              & Aloção múltipla   & 0                           & 0                         \\
    Sala              & Capacidade        & 0                           & 0                         \\
    Disciplinas       & Inadequação       & 0                           & 0                         \\
    \hline
  \end{tabular}
\end{table}

\subsection{Comparativo das tabelas horárias}

Considerando a parcial recorrência de disciplinas de mesma paridade, dispõe-se aqui uma comparação entre as tabelas horárias geradas para o curso de Ciência da Computação e as grades finais dos dois últimos anos (2022.2, disposto na \autoref{fig:grade-horaria-cc-2022.2} e 2023.2 disposto na \autoref{fig:grade-horaria-cc-2023.2}).

\begin{MyCenteredFigure}
  \caption{Grade horária fictícia de Ciência da Computação para 2022.2}
  \label{fig:grade-horaria-cc-2022.2}
\end{MyCenteredFigure}

\begin{MyCenteredFigure}
  \caption{Grade horária fictícia de Ciência da Computação para 2023.2}
  \label{fig:grade-horaria-cc-2023.2}
\end{MyCenteredFigure}

Como forma de comparação, na (\autoref{fig:grade-horaria-cc-2024.2}) está disposta a tabela horária gerada exibindo exclusivamente as disciplinas oferecidas para o curso de Ciência da Computação.

\begin{MyCenteredFigure}
  \caption{Grade horária fictícia de Ciência da Computação para 2024.2}
  \label{fig:grade-horaria-cc-2024.2}
\end{MyCenteredFigure}

Nota-se que devido ao tamanho da imagem, de toda a grade do CCT, não é viável a comparação visual entre as grades horárias. visto que foram ofertadas em torno de 279 disciplinas, sendo 30 delas para o curso de Ciência da Computação. Assim, devido ao voluma, a apresentação gráfica da grade horária do CCT foi omitida.

\section{Análise dos resultados}

Assim como estimado, a etapa mais demorada foi a alocação das turmas do CCT, que demandou um tempo considerável para a sua realização. As demais etapas foram realizadas em um tempo de acordo com o esperado.

Também esperado foi a ocorrência de conflitos, que foram resolvidos sem maiores problemas e em um espaço de tempo agradável.

\section{Conclusões}
